\section{Introduction}\label{sec:intro}
\subsection{Overview}\label{subsec:overview}
Real GDP is a prominent measure of output due to its simplicity, interpretability, and effectiveness in communicating the overall activity of an economy. 
Alongside academic settings, Real GDP has retained its relevance in politics as one of the go-to measures for public communications regarding economic growth and prosperity. 
As per its wide-spread adoption, the measure has accumulated a plethora of literature referencing it from a multitude of perspectives.

\vspace{\baselineskip}

This paper aims to show that empirical observations of Real GDP in the U.S. economy can be modelled optimally with a \textbf{Gauss\rule[0.5ex]{0.5em}{0.7pt}Markov} compliant $ARX(p, q)$ specification where the exogenous contributors are theoretical descriptors of output.

\subsection{Hypothesis}\label{subsec:hypothesis}
In the history of economic equilibria, multiple models have proposed mathematically defined
relationships between Real GDP and many indicators. From the Keynesian approach, two of said
indicators are:

\begin{itemize}
    \item Price levels (CPI) through the \textbf{Aggregate Demand Curve}
    \item Unemployment rates through \textbf{Okun's Law}
\end{itemize}

The formal hypothesis of this paper is that an $ARX(p, q)$ representation of Real GDP, incorporating CPI and unemployment:


\begin{equation}
    \label{eq:model_eq}
    ARX_y(p, \ q) = C + \sum_{i=1}^p \beta_i y_{t-i} + \sum_{j=1}^q \big{\lbrack}\phi_j CPI_{t-j} + \gamma_j u_{t-j}\big{\rbrack} + \varepsilon_t
\end{equation}

\noindent yields OLS parameter estimates that satisfy the \textbf{Gauss\rule[0.5ex]{0.5em}{0.7pt}}Markov conditions for time-series estimators.

\subsection{Theoretical Background}\label{subsec:theory}
We consider two equations to base our selection of exogenous variables.
First, the \textbf{Phillips Curve} describes the relationship between inflation and unemployment.
It can be expressed in terms of expected inflation and the unemployment gap as follows:
\begin{equation}
    \pi_t = \mathbb{E}_t[\pi_{t+1}] - \alpha (u_t - u^\star) + \varepsilon_t
    \label{eq:philips_gap}
\end{equation}
where $\pi_t$ is the inflation rate at time $t$, $\mathbb{E}_t[\pi_{t+1}]$ is the expected inflation rate at time $t+1$, $\bar{u}_t$ is the unemployment gap, $\alpha$ is a positive constant, and $\varepsilon_t$ is a random error term.
We're interested in the partial contributions of the deterministic terms, therefore $\varepsilon_t$ will be omitted from equations in this section.
First, we transform $\mathbb{E}_t[\pi_{t+1}]$ to be expressed in terms of price levels:
\begin{equation}
    \begin{aligned}
        \mathbb{E}_t[\pi_{t+1}] &= \mathbb{E}_t\bigg\lbrack \frac{P_{t+1} - P_t}{P_t}\bigg\rbrack \\
        &= \frac{\mathbb{E}_t[P_{t+1}] - P_t}{P_t}
    \end{aligned}
    \label{eq:expected_inflation}
\end{equation}

Substituting this back into Equation~\ref{eq:philips_gap} gives:
\begin{equation}
    \pi_t = \frac{\mathbb{E}_t[P_{t+1}] - P_t}{P_t} - \alpha (u_t - u^\star)
    \label{eq:philips_price}
\end{equation}
This adjustment transforms the Philips Curve to have computable partial contributions from price levels at time $t$ and expectations of time $t+1$.
Next, we consider \textbf{Okun's Law}, which describes the relationship between unemployment and output (Real GDP):
\begin{equation}
    \frac{(Y-Y^\star)}{Y^\star} = -\gamma(u_t - u^\star)
    \label{eq:okuns_law}
\end{equation}
where $Y$ is the actual output, $Y^\star$ is the potential output, $\gamma$ is a positive constant, and $(u_t - u^\star)$ is the unemployment gap.
Okun's Law does not require transformations to have a computable partial contribution from unemployment.

\vspace{\baselineskip}
Using the above equations, we can compute unit-change in output per exogenous variable (partial derivatives), and percent-change in output per exogenous variable (elasticities).
Starting with price levels, we have:
\begin{equation}
    \begin{aligned}
        \frac{\delta \pi_t}{\delta P_t} &= -\frac{\mathbb{E}_t[P_{t+1}]}{P_t^2} \\
        \epsilon_{\pi_t, P_t} &= \frac{\mathbb{E}_t[P_{t+1}]}{P_t(P_t\alpha(u_t - u^\star)) + P_t - \mathbb{E}_t[P_{t+1}]}
    \end{aligned}
    \label{eq:partial_price}
\end{equation}


Next, for unemployment, we have:
\begin{equation}
    \begin{aligned}
        \frac{\delta \big(\frac{\bar{Y}}{Y^\star}\big)}{\delta u_t} &= -\gamma \\
        \epsilon_{\frac{\bar{Y}}{Y^\star}, u_t} &= \frac{1}{u_t - u^\star}
    \end{aligned}
    \label{eq:partial_unemp}
\end{equation}

A simple yet fundamental observation is that both exogenous variables have non-zero contributions to output.
However, the above defined relationships of $P_t$ are relative to inflation $\pi_t$ instead of output $Y$.

\subsection{Literature Review}\label{subsec:litreview}
The literature review in this paper is aimed to justify two main choices made for our methodology and selection of variables:
\begin{itemize}
    \item Selection of the exogenous variable set
    \item Usage of an $ARX$ model
\end{itemize}

The theoretical relationships between exogenous variables and output were displayed in Section~\ref{subsec:theory}. 
In addition to said demonstration, there are numerous papers investigating these relationships.
Primarily, \enquote{The Relationship between GDP and Unemployment Rate in the U.S.} (\cite{mandel_relationship_2019}) employs a very similarly structured model to conduct correlation analysis.
The paper demonstrates the existence of significant correlations $\rho_{Y, CPI}$ and $\rho_{Y, u}$ (among other variables) through multiple regression analysis.
However, an important note regarding their application is the usage of $CPI_{t-0}$ and $u_{t-2}$ which do not align the orders selected in our paper.\footnote{Refer to Section~\ref{subsubsec:ar_order} for the order selection of this paper.}

\vspace{\baselineskip}

Another relevant paper, \enquote{GDP Forecasting: Machine Learning, Linear or Autoregression?} (\cite{maccarrone_gdp_2021}), compares the performance of various simple regression models and machine learning algorithms in forecasting GDP.
The authors conclude that $ARX$ in their specification is the best model in \enquote{one-step-ahead} and comparable to the best in \enquote{multi-step-ahead} forecasts.
This paper confirms the intuitively autocorrelated nature of GDP, and the fact that exogenous variables have the ability to improve fit quality.

% Numerous of earlier research show that price levels (CPI) and unemployment rates (u) are predictors of GDP. 
% A multiple regression study conducted correlation analysis using a model containing CPI and $u$ (Among INDRO and Personal Income) against Real GDP and confirmed the existence of such correlation. (\cite{mandel_relationship_2019})



% They estimated the models of link between real GDP per capita and unemployment rate over eight developed countries.
% Indifferent from previous research (\cite{mandel_relationship_2019}), they pointed out the real GDP per capita is necessary to 
% present the real economic growth while the price levels and unemployment rate are both real economic indicators.
% It shows the accurate prediction of the dynamics of independent variable since 1960s. 
% In addition, (\cite{rouksar-dussoyea_economic_2017}) research highlight the importance of both independent variables (CPI and unemployment rate) in determining the real GDP.
% Matching the result in (\cite{kitov_link_2021}), showing significant relationship with dependent variable (GDP). 

% \vspace{\baselineskip}

% In (\cite{chen_development_2025}) study on GDP forecasting support the use of two key predictors (CPI and unemployment rate). In their model,
% used two independent variables (CPI and unemployment rate) to predict real GPD per capita and showed high accuracy even in advanced models.
% Together with (\cite{mandel_relationship_2019}) and (\cite{kitov_link_2021}) demonstrated short-term economics flucatoins and long-term structural conditions.
% They also pointed out both traditonal time-series models and morden deep learning models, indicators (CPI and unemployment rate) are the fundamental drivers of real economics growth.
% Theoretically, it supports the hypothesis proposed with ARX model in this paper.

% \subsubsection{Empirical Research}\label{subsubsec:empirical_research}
% In (\cite{tang_application_2020}) study, they're approaching macronomic forecasting by nonlinear autogressive with exogenous inputs model (NARX).
% It shows the autogression structure with exogenous inputs (CPI and unemployment rate) match the real economics activity. 
% While they found adding exogenous variable like economic key indciator can improve the prediction.
% Even though the NARX model is different from ARX model, we can see the economic growth jsut like dynamic system with past value and respond to extenral variable like price levels and unemployment rate.







% In (\cite{mandel_relationship_2019} \&~\cite{kitov_link_2021}), we can see
% both apply Breusch--Pagan test and Variance Inflation Factor.
% Among independent variables, Variance Inflaiton Factor (VIF) is widely used in showing the regression cofficients are reliable and stable.
% Same for Breusch--Pagan test, it is a common method to detect the hetroscedasticity in regreession model. 
% Moverover, (\cite{granville_tunnicliffe_pdf_2025}) show just pure autogressive models often can't capture the full dynamics of time series due to external forces.
% All evidences and diagontic tests show the nescessity of exogenous variable for well fitted model.