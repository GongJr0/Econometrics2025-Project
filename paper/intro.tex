\section{Introduction}\label{sec:intro}
\subsection{Overview}\label{subsec:overview}
Real GDP is a prominent measure of output due to its simplicity, interpretability, and effectiveness in communicating the overall activity of an economy. 
Alongside academic settings, Real GDP has retained its relevance in politics as one of the go-to measures in public communications regarding economic growth and prosperity. 
As per its wide-spread adoption, the measure has accumulated a plethora of literature referencing it from a multitude of perspectives.

\vspace{\baselineskip}

This paper aims to show that empirical observations of Real GDP in the U.S. economy can be modelled optimally with a \textbf{Gauss\rule[0.5ex]{0.5em}{0.7pt}Markov} compliant $ARX(p, q)$ specification where the exogenous contributors are theoretical descriptors of output.

\subsection{Hypothesis}\label{subsec:hypothesis}
In the history of economic equilibria, multiple models have proposed mathematically defined
relationships between Real GDP and many indicators. From the Keynesian approach, two of said
indicators are:

\begin{itemize}
    \item Price levels (CPI) through the \textbf{Aggregate Demand Curve}
    \item Unemployment rates through \textbf{Okun's Law}
\end{itemize}

The formal hypothesis of this paper is that an $ARX(p, q)$ representation of Real GDP, incorporating CPI and unemployment:


\begin{equation}
    \label{eq:model_eq}
    ARX_y(p, \ q) = C + \sum_{i=1}^p \beta_i y_{t-i} + \sum_{j=1}^q \big{\lbrack}\phi_j CPI_{t-j} + \gamma_j u_{t-j}\big{\rbrack} + \varepsilon_t
\end{equation}

\noindent yields OLS parameter estimates that satisfy the \textbf{Gauss\rule[0.5ex]{0.5em}{0.7pt}}Markov conditions for time-series estimators.

\subsection{Literature Review}\label{subsec:litreview}
\dots