\section{Results}\label{sec:results}
\subsection{Design Matrix}\label{subsubsec:design_matrix}
First step of the design matrix creation is the identification of optimal $\{p, \ q\}$ parameters for the model.

\subsubsection{Autoregressive Order}\label{subsubsec:ar_order}
Application of ACF yielded a very common pattern of monotonically decreasing correlations, not outlining any specific lag order as superior.
Looking for more conclusive insights, PACF was applied.

\begin{figure}[ht]
    \centering
    \includegraphics[width=\textwidth]{fig/acf_pacf}
    \caption{ACF and PACF plots of the dependent (target) variable.}
    \label{fig:acf_pacf}
\end{figure}

Figure~\ref{fig:acf_pacf} clearly shows the aforementioned decay in ACF\@. 
Conversely, PACF shows lag-1 is the only entry with actual partial contribution into the descriptive structure of $y$.\footnote{The first spike in both ACF and PACF refers to lag-0 (the variable itself) which is equal to 1 by definition.}
Interestingly, the small spikes around lag-20 of the PACF plot (corresponding to 20 quarters or 5 years) indicate that the COVID-19 shock still has a minor effect on the 2025 Real GDP\@.

\vspace{\baselineskip}

In any case, the visualization of the autocorrelation structure presents strong evidence towards the $p=1$ selection, which is used for the model generation in this paper.

\subsubsection{Exogenous Order}\label{subsubsec:exog_order}
The exogenous order selection process started with the visualization of the simple correlation vector $y$ and the lags of exogenous variables.

\begin{table}[ht]
    \centering
    \begin{tabular}{|c|c|c|}
        \hline
        $\mathbf{i_{lag}}$ & $\mathbf{corr(y, \ CPI_{t-i})}$ & $\mathbf{corr(y, \ u_{t-i})}$ \\
        \hline
        0 & 0.989 & -0.221 \\
        1 & 0.988 & -0.255 \\
        2 & 0.988 & -0.221 \\
        3 & 0.988 & -0.190 \\
        4 & 0.988 & -0.236 \\
        5 & 0.988 & -0.205 \\
        \hline
    \end{tabular}
    \caption{Pearson correlation coefficients between the dependent variable and lags of exogenous variables.}
    \label{tab:exog_corr_matrix}
\end{table}

Inspecting the table of coefficients, we that $CPI$ consistently stays at a coefficient of 0.988\@. This indicates that the autoregressive process of $CPI$ is likely very strong, and related to the autoregressive process explaining $y$.
On the other hand, $u$ shows a more reasonable set of coefficients. Although coefficients of $u$ are also relatively stable, they reside in the reasonable range of $[-0.255, \ -0.190]$.
However, a lack of decay or noticeable regime changes in the matrices leaves this simple test inconclusive.
This inconclusiveness was expected, and the CCF plots of whitened variables were examined for clearer insights.

\begin{figure}[ht]
    \centering
    \includegraphics[width=\textwidth]{fig/ccf}
    \caption{CCF plots of the whitened exogenous variables and the dependent variable.}
    \label{fig:ccf}
\end{figure}

Although not as clear as the PACF results, Figure~\ref{fig:ccf} carries much stronger insights into the cross-correlation structure compared to raw correlation coefficients.
$CPI$ shows several spikes above the significance bounds with notable lags being $\{0, 45, 47\}$.
For $u$, we observe that lag-0 is the only significant entry.
Based on the observations, lag-0 is the only selection that ensures significant lags of both exogenous variables are included.
Therefore, the cross-correlation analysis yields $q=0$ to be the optimal selection.

\vspace{\baselineskip}
Using $q=0$ introduces the problem of \textbf{data leakage}, as current quarter values of exogenous variables would not be available at the time of estimation.
This results in a design choice of either using the closest available selection in time (i.e. lag-1) or in significance (selection of lag-45 or lag-47 for $CPI$).
Since introducing 45 or 47 more coefficients per variable would essentially guarantee overfitting, the final decision was in favor of using $q=1$.\footnote{Further regularization techniques, or using sparse lagged estimators are indeed possible but are deemed outside the scope of this paper.}

\subsection{Model Fit}\label{subsubsec:model_fit}
\dots

\subsection{Residual Analysis}\label{subsubsec:residual_analysis}
\dots

\subsection{Gauss--Markov Conditions}\label{subsubsec:gm_conditions}
\dots

\subsection{Residual Distribution}\label{subsubsec:residual_distribution}
\dots
