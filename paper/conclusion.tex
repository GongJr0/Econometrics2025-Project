\section{Conclusion}\label{sec:conclusion}

With the help of normalization of variables, we were able to construct a well-behaved $ARX(1, 1)$ model that, against expectations, showed no evidence against BLUE.
Although the model showed signs of overfitting in-sample, the statistical tests and equality of variable distributions support the conclusion of an extremely interpretable estimator where coefficients may show connections to empirical elasticities of the used Keynesian regressors.

\vspace{\baselineskip}

An unexpected finding was made with the unemployment coefficient, but the model as a whole shows enough stability and in-sample goodness-of-fit to warrant further comparison of empirical coefficients to economic theory.
Moreover, we were able to display the use new and developing statistical tests and adapt to the unique challenges brought by the time-series context.

\vspace{\baselineskip}
All in all, we believe that out model provides a decent starting point for further research and analysis.
While the limited out-of-sample testing did not yield results that immediately display predictive instability, we strongly believe that as more data becomes available, the model will exhibit the well-known symptoms of overfitting.
Therefore, we want to stress that the idea behind the creation of this model is to explore where theory and empirical evidence meet, and we encourage future researchers to build upon our findings.