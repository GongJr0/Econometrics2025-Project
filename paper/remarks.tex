\section{Remarks}\label{sec:remarks}

\subsection{Remark 1: VIF Application on a Reduced Design Matrix}\label{subsec:remark_vif}
Although VIF just a measure and has no constraints on its application, the conventional use is to check for multicollinearity among the complete $X$ matrix.
In models with AR components, this application of VIF leads to two issues (assuming the dependent variable is actually autocorrelated):

\begin{itemize}
    \item The AR components inside $X$ will create processes equivalent to the an AR estimator of $y$. 
    To demonstrate, let's assume a relationship $y_t \sim y_{t-1}$ exists in an $AR(2)$ model. 
    As shifting the variable in time does not change the assumed relationship, we also have $y_{t-1} \sim y_{t-2}$.
    As a consequence, the design matrix (which is composed of $y_{t-1}$ and $y_{t-2}$) will exactly replicate an $AR(1)$ estimator of $y$.
    With lags selected to maximize $R^2_y$, we end up also maximizing the $R^2$ in the VIF calculation.

    \item The point above extends to exogenous variables as well.
    Again, assuming $y_t \sim x_{t-1}$ exists, we also have $y_{t-1} \sim x_{t-2}$.
    Although $y_{t-1} \sim x_{t-2}$ never directly occurs, the design matrix in this case produces $y_{t-1} \sim x_{t-1}$.
    This is not an exact replication of the designed relationship, but by autocorrelation of $y$, it is a very safe assumption that $y_{t-1} \sim x_{t-1}$ will produce an $R^2$ comparable to the original estimator.
    As a result, we will again be maximizing the $R^2$ in the VIF calculation if we're constructing the model to explain the dependent variable as best as it can.

\end{itemize}

The points discussed show the well-known fact that VIF is not designed for these scenarios.
Therefore, many studies, alongside this paper, elect to use VIF on a reduced $X$ matrix stripped of AR components. (\cite{niu_impact_2022} \& \cite{karlstrom_data-driven_2023})

\subsection{Remark 2: Time-Series Application of KS Test}\label{subsec:remark_ks}

The i.i.d.\@ assumption of the classic KS test was addressed by Præstgaard with a bootstrapping approach.
Yet, the examples, lemmas, and theorems presented in the paper were derived from row-exchangeable samples.

\vspace{\baselineskip}

With row-exchangeability, Præstgaard was able to apply standard bootstrap resampling while preserving the continuity of the empirical distribution.
However, time-series data inherently requires the row-structure to be preserved; hence rendering the standard bootstrap inappropriate.
To preserve the time-dependent dynamics, our methodology elected to use a \textbf{Block Bootstrap} approach. 
This specific adaptation is not documented nor theoretically justified in Præstgaard's work; consequently, while we regard it as a reasonable and practically reliable extension for the purposes of this study, it remains an empirical choice that invites further theoretical investigation.

\vspace{\baselineskip}

A block bootstrap 1-sample KS test was derived in a recent preprint paper (\cite{chandy_nonparametric_2025}) and applications of bootstrap techniques in KS style tests were demonstrated in multiple studies. (\cite{psaradakis_bootstrap_2003} \& \cite{wylupek_nonparametric_2023})
These works place bootstrap KS tests on a more solid theoretical footing, even though our specific use case is undocumented.