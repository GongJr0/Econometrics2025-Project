\section{Data}\label{sec:data}
The proposed ARX model relies on two exogenous variables alongside the autoregressive target.
All three variables are sourced from the Federal Reserve Bank of St. Louis database “FRED”. 
The implementation uses quarterly data from 1990Q1 to 2025Q1 and the following FRED series:

\begin{itemize}
    \item Real GDP ($y$): \enquote{\href{https://fred.stlouisfed.org/series/GDPC1}{GDPC1}}
    \item CPI ($CPI$): \enquote{\href{https://fred.stlouisfed.org/series/CPIAUCSL}{CPIAUCSL}}
    \item Unemployment Rate ($u$): \enquote{\href{https://fred.stlouisfed.org/series/UNRATE}{UNRATE}}
\end{itemize}

\noindent NOTE: All series are seasonally adjusted by FRED.

\subsection{Series Characteristics}\label{subsec:series_characteristics}
\subsubsection{Real GDP}\label{subsubsec: rgdp}
FRED's Real GDP series is measured in chained 2017 dollars (Bn). 
The series is updated quarterly, which matches the frequency of the model natively.

\vspace{\baselineskip}
\begin{figure}[ht]
    \centering
\begin{minipage}[t]{0.45\textwidth}
    \flushright
    \begin{tabular}{lc}
        \toprule
        Statistic & Value \\
        \midrule
        $\bar{X}$ & 16,213.22 \\
        $S_X$ & 3,837.33 \\
        $\operatorname{Skew}(X)$ & 0.063 \\
        $\operatorname{Kurt}(X)$ & -0.945 \\
        \bottomrule
    \end{tabular}
    \label{tab:rgdp_stats}
\end{minipage}
\hfill
\begin{minipage}[t]{0.45\textwidth}
    \flushleft
    \begin{tabular}{lc}
        \toprule
        Quantile & Value \\
        \midrule
        0.00 & 9,951.91 \\
        0.25 & 13,191.67 \\
        0.50 & 16,485.35 \\
        0.75 & 19,062.71 \\
        1.00 & 23,586.54 \\
        \bottomrule
    \end{tabular}
    \label{tab:rgdp_quants}
\end{minipage}
\caption{Descriptive statistics and quantiles of Real GDP.}
\end{figure}

\subsubsection{CPI}\label{subsubsec: cpi}
The CPI series of choice for the model records the US-wide city average CPI for all urban consumers. 
The series is expressed as a chain index with base period 1982-1984=100. 
Entries are recorded monthly, and quarter-end observations are used to align with the Real GDP series.

\vspace{\baselineskip}
\begin{figure}[H]
    \centering
\begin{minipage}[t]{0.45\textwidth}
    \flushright
    \begin{tabular}{lc}
        \toprule
        Statistic & Value \\
        \midrule
        $\bar{X}$ & 207.32 \\
        $S_X$ & 49.10 \\
        $\operatorname{Skew}(X)$ & 0.343 \\
        $\operatorname{Kurt}(X)$ & -0.708 \\
        \bottomrule
    \end{tabular}
    \label{tab:cpi_stats}
\end{minipage}
\hfill
\begin{minipage}[t]{0.45\textwidth}
    \flushleft
    \begin{tabular}{lc}
        \toprule
        Quantile & Value \\
        \midrule
        0.00 & 127.50 \\
        0.25 & 163.90 \\
        0.50 & 207.60 \\
        0.75 & 238.99 \\
        1.00 & 319.09 \\
        \bottomrule
    \end{tabular}
    \label{tab:cpi_quants}
\end{minipage}
\caption{Descriptive statistics and quantiles of CPI.}
\end{figure}

\subsubsection{Unemployment Rate}\label{subsubsec: unemp}
Unemployment rate in FRED is recorded as percentages with a monthly frequency. 
Similar to CPI, quarter-end observations are used to match the frequencies among the series.

\vspace{\baselineskip}
\begin{figure}[ht]
    \centering
\begin{minipage}[t]{0.45\textwidth}
    \flushright
    \begin{tabular}{lc}
        \toprule
        Statistic & Value \\
        \midrule
        $\bar{X}$ & 5.71 \\
        $S_X$ & 1.75 \\
        $\operatorname{Skew}(X)$ & 1.249 \\
        $\operatorname{Kurt}(X)$ & 2.068 \\
        \bottomrule
    \end{tabular}
    \label{tab:u_stats}
\end{minipage}
\hfill
\begin{minipage}[t]{0.45\textwidth}
    \flushleft
    \begin{tabular}{lc}
        \toprule
        Quantile & Value \\
        \midrule
        0.00 & 3.40 \\
        0.25 & 4.40 \\
        0.50 & 5.40 \\
        0.75 & 6.60 \\
        1.00 & 14.80 \\
        \bottomrule
    \end{tabular}
    \label{tab:u_quants}
\end{minipage}
\caption{Descriptive statistics and quantiles of Unemployment Rate.}
\end{figure}